\documentclass{article}
\usepackage[a4paper,margin=1in]{geometry}
\usepackage{graphicx}
\usepackage{hyperref}

\title{KiwiICP Interface: Redefining AI-Powered Educational Assistance}
\author{Team \textit{Plumber}}
\date{}

\begin{document}

\maketitle

\section*{Abstract}
KiwiICP is an innovative educational assistant built on the Interactive Course Platform (ICP). It provides contextual support to students by following their progress on course slides and fostering an understanding of problem-solving approaches. This proposal presents enhancements aimed at making KiwiICP more efficient, personalized, and integrative, focusing on course progress tracking, learning management system (LMS) integration, self-learning support, teacher tools, and adjustable AI proficiency.

\section{Introduction}
Modern education demands tools that enhance learning through real-time, contextual guidance while encouraging problem-solving skills. KiwiICP fulfills this role by leveraging AI to assist students in navigating course content, solving assignments, and engaging with study materials.

However, current limitations such as scattered chat histories and lack of system integration restrict its potential. Our proposal seeks to refine KiwiICP, creating a more robust, connected, and user-friendly experience for students and educators.

\section{Proposed Enhancements}
\subsection{Course-Based Chat Histories}
Unlike traditional chat systems with scattered conversation records, KiwiICP will organize chat histories based on the courses enrolled. This enhancement will:
\begin{itemize}
    \item Track progress and context across the semester.
    \item Provide course-specific suggestions and solutions.
    \item Automatically review prerequisite topics if students encounter difficulties.
\end{itemize}

\subsection{LMS Integration (Brightspace and GradeScope)}
Connecting KiwiICP with LMS platforms ensures a comprehensive tracking and assistance framework:
\begin{itemize}
    \item Link announcements, grades, and assignments directly to KiwiICP.
    \item Alert students to review related topics when errors occur in assignments.
    \item Send push notifications for upcoming deadlines and updates.
\end{itemize}

\subsection{Self-Learning Support}
Beyond classroom learning, KiwiICP will:
\begin{itemize}
    \item Integrate external tutorials and resources.
    \item Answer queries related to self-paced study.
    \item Offer guided tutorials to bridge gaps in understanding.
\end{itemize}

\subsection{Advanced Teacher Tools}
Teachers will have access to enhanced tools that allow:
\begin{itemize}
    \item Real-time updates to course content.
    \item Advanced analytics for student performance.
    \item Identification of knowledge gaps and suggestions for remedial actions.
\end{itemize}

\subsection{Adjustable Proficiency Levels}
The "Zoom-In/Zoom-Out" feature will allow users to:
\begin{itemize}
    \item Adjust KiwiICP's proficiency level based on individual needs.
    \item Explore advanced content or simplify explanations for foundational understanding.
    \item Tailor assistance to different levels of academic proficiency.
\end{itemize}

\section{Implementation Details}
\subsection{Technical Design}
The proposed enhancements rely on:
\begin{itemize}
    \item API-based integration for LMS platforms.
    \item Modular architecture for scalability.
    \item AI models fine-tuned for contextual learning and feedback loops.
\end{itemize}

\subsection{Feasibility Analysis}
Our design aligns with current educational technologies and addresses specific pain points identified during pilot studies. Key insights from NYU Shanghai highlight:
\begin{itemize}
    \item The need for seamless integration with existing learning tools.
    \item Demand for personalized, contextualized assistance.
\end{itemize}

\section{Benefits of the Enhanced KiwiICP}
\begin{itemize}
    \item Increased engagement and retention through contextual learning.
    \item Bridging classroom learning and self-study with a unified assistant.
    \item Empowering educators with data-driven teaching strategies.
\end{itemize}

\section{Conclusion}
The proposed refinements to KiwiICP will transform it into a comprehensive educational assistant that empowers students and teachers alike. By combining personalization, LMS integration, and adaptive features, KiwiICP sets a new standard for AI in education.


\end{document}